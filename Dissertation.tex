
\documentclass{gist}

\def\putepsf#1{\centering \parbox{14cm}{\epsfxsize = 14cm \epsfbox{#1}}}
\usepackage{graphicx}

\usepackage{amsmath, amssymb}

% \usepackage{algorithm}
% \usepackage{amsfonts}
% \usepackage{algorithmic}
\AtBeginEnvironment{algorithm}{\setstretch{1.0}}
\usepackage[lofdepth=1]{subfig}
\usepackage[rightcaption]{sidecap}
\usepackage{multirow}
\usepackage{makecell}
% \usepackage{array}

% yoon
\usepackage[ruled,vlined,linesnumbered]{algorithm2e}
\usepackage{etoolbox}
\usepackage{url}

\newcommand\etal{\textit{et~al.\ }}
\newcommand\eg{\textit{e.g.\ }}
\newcommand\ie{\textrm{i.e.\ }}
\newcommand\R{\mathbb{R}}
% yoon

% \usepackage{theorem}
\usepackage{setspace}
\usepackage{appendix}
\usepackage{cite}
\usepackage{csquotes}

\usepackage[colorlinks=true,linkcolor=black,citecolor=black]{hyperref}
% \usepackage[]{hyperref}
% \hypersetup{
%     % bookmarksnumbered=true,
%     %   linkbordercolor=blue,
%     %   filebordercolor=red,
%     %   citebordercolor = green,      
%     %   urlbordercolor=red,
%       %
%       colorlinks=false
%       % linkcolor=blue,
%       % filecolor=blue,
%       % citecolor = blue,      
%       % urlcolor=cyan,
% }

\usepackage{bookmark}
\bookmarksetup{numbered}

\makeatletter
\bookmarksetup{%
  addtohook={%
    \ifnum\toclevel@chapter=\bookmarkget{level}\relax
      \renewcommand*{\numberline}[1]{CHAPTER #1 }%
    \fi
  },
}
\makeatother

\usepackage{cleveref, autonum}

\usepackage{amsthm}
\newtheorem{theorem}{Theorem}[section]
\newtheorem{lemma}{Lemma}[section]
\newtheorem{corollary}{Corollary}[section]
\newtheorem{proposition}{Proposition}[section]
\newtheorem{assumption}{Assumption}[section]
\theoremstyle{remark}
\newtheorem{remark}{Remark}[section]
\theoremstyle{definition}
\newtheorem{definition}{Definition}[section]

%-----------------------------------------------------------------------
% Department code list
% IC - Information and Communications
% IM - Information and Mechatronics
% EC - Electric Engineering and Computer Science
% MS - Materials Science and Engineering
% ME - Mechanical and Robotics Engineering
% AI - Artificial Intelligence Graduate School
% EN - Earth Science and Environmental Engineering
% LS - Life Science
% PH - Physics and Photon Science
% CH - Chemistry
% NA - Nanobio Materials and Electronics
% MD - Biomedical Science and Engineering
% ET - Integrated Technology 에너지 융합 학제
% CT - Integrated Technology 문화기술 융합 학제
% RT - Integrated Technology 지능로봇프로그램

% Department code
%\code{{BS/}{EC}}
%\code{{MS/}{EC}} % 학위와 소속 코드, 초록 페이지에 나타남
\code{{MS/}{AI}}
%-----------------------------------------------------------------------
% Thesis title in English
% Insert \titlebreak where lines are to be separated.  Do not use the LaTeX command '\\'.

%\etitle{Study on Image Segmentation and Action Recognition in Computer Vision}

\etitle{State-wise Safety in Autonomous Driving \titlebreak via Lagrangian-based \titlebreak Constrained Reinforcement Learning}

%\etitle{Study on Segmentation and Recognition for an Image Understanding}
%\etitle{Genetic Programming: \titlebreak New Optimization Tools for Real-World Applications}

%-----------------------------------------------------------------------
% Thesis title in Korean
% Insert \titlebreak where lines are to be separated.  Do not use the LaTeX command '\\'.
%\begin{spacing}{2.0}
\ktitle{라그랑지안 기반 제약 강화학습을 이용한 자율주행에서의 상태별 안전성 고려}
%\end{spacing}
%-----------------------------------------------------------------------
% Advisor's name in English without a position such as 'Prof.'.
\advisor{Ue-Hwan Kim}

%-----------------------------------------------------------------------
% Advisor's name in Korean without a position such as 'Prof.'.
\kadvisor{김의환}

%-----------------------------------------------------------------------
% Co-advisor's name in English
% In case there is no co-advisor, comment out the following line with a "%" in the front.
%\coadvisor{My Co-advisor}

%-----------------------------------------------------------------------
% Name of the author in English
\ename{Minseok Seo}

%-----------------------------------------------------------------------
% Name of the author in Korean seperated with '{}'.
\kname{{}{}{}{}{서}{민}{석}} % 한글 이름 7글자까지 가능, 오른쪽 끝에 맞춰서 입력

%-----------------------------------------------------------------------
% Student ID of the author
\studentid{20231115}

%-----------------------------------------------------------------------
% The year of graduation (ex. 1999)
\coveryear{2025}

%-----------------------------------------------------------------------
% The date signed by the advisor.  The first is the month, second the date, and third the year.
\advisorsigndate{June}{20}{2025}

%-----------------------------------------------------------------------
% The date signed by the referees.  The first is the month, second the date, and third the year.
\refereesigndate{June}{20}{2025}

%-----------------------------------------------------------------------
% Names of the referees in English
% For Master's thesis, input the names of the three referees (refereeA thru referee C) in full.
% For Ph.D thesis, input the names of the five referees (refereeA thru referee E) in full.
% For most cases, refereeA is the same as the advisor.

\refereeA{Prof. Ue-Hwan Kim.}
\refereeB{Prof. Kyung Joong Kim.}
\refereeC{Prof. Kyunghwan Choi.}
% \refereeD{Prof. fourth prof.}
% \refereeE{Prof. fifth prof.}

%-----------------------------------------------------------------------
% This is the beginning of the thesis.

\dedication{
Dedicated to my family.
}

\begin{document}
%-----------------------------------------------------------------------
% Abstract of the thesis in English.
% Insert the abstract between \begin{eabstract} and \end{eabstract}.
% You can either write the abstract directly here or import a file using the \input command.

%%%%%%%%%%%%%%%%%%%%%%%%%%%%%%%%%%%%%%%%%%
% Abstract by English
%%%%%%%%%%%%%%%%%%%%%%%%%%%%%%%%%%%%%%%%%%

\begin{eabstract}
\begin{spacing}{2.0} % double spacing

%%%%%%%%%%%%%%%%%%%%%%%%%%%%%%%%%%%%%%
% Abstract of the thesis in English
%%%%%%%%%%%%%%%%%%%%%%%%%%%%%%%%%%%%%%

The abstract will be located here.

\end{spacing}
\end{eabstract}

%-----------------------------------------------------------------------
% Abstract of the thesis in Korean.
% Insert the abstract between \begin{kabstract} and \end{kabstract}.
% You can either write the abstract directly here or import a file using the \input command.

%%%%%%%%%%%%%%%%%%%%%%%%%%%%%%%%%%%%%%%%%%
% Abstract by Korean
%%%%%%%%%%%%%%%%%%%%%%%%%%%%%%%%%%%%%%%%%%

\begin{kabstract}
\begin{spacing}{2.0} % double spacing

%국문초록
%%%%%%%%%%%%%%%%%%%%%%%%%%%%%%%%%%%%%%
% Abstract of the thesis in Korean
%%%%%%%%%%%%%%%%%%%%%%%%%%%%%%%%%%%%%%

자율주행 시스템의 실제 적용을 위해서는 높은 안정성과 적응성이 요구된다.
이에 따라, 시행착오를 통해 주행 전략을 학습하며 발전시키는 심층 강화 학습(Deep Reinforcement Learning, DRL)이 주목받고 있다.
하지만 강화 학습은 본질적으로 보상을 극대화하는 방향으로 정책을 학습하기 때문에, 학습 후에도 안전하지 않거나 비정상적인 행동을 할 가능성을 완전히 배제하기 어렵다.
이러한 한계를 해결하기 위해, 정책 학습 시 안정성과 성능 간의 균형을 도모하는 제약 강화 학습(Constrained Reinforcement Learning, CRL)이 제안되었다.
제약 강화 학습은 기댓값 기반 누적 비용 형태의 제약 조건을 만족하도록 정책을 학습하지만, 각 상태에서의 제약 조건 충족 여부를 고려하지 않아 상태별 안정성을 보장하기 어렵다.
본 논문에서는 제약 강화 학습의 한 방식인 라그랑지안 기반의 방법을 확장하여, 상태별 라그랑주 승수를 추정함으로써 정책이 상태별 안정성을 고려하도록 한다.
또한 제안한 방법을 OpenAI의 시뮬레이션 환경인 Safety Gym을 통해 기존 라그랑지안 기반의 방법들과 비교하여 검증하였다.

\end{spacing}
\end{kabstract}


%-----------------------------------------------------------------------
% Table of contents, list of tables and list of figures.
% Use the \makecontents command to automatically generate the table of content
\makecontents

% In case there is no table, comment out the following line.
% \listtables

% In case there is no figure, comment out the following line.
\listfigures

% In case there is no algorithm, comment out the following line.
% \listalgorithms

%-----------------------------------------------------------------------
% Input the thesis files written in LaTeX.
% The \begin{document} command is not necessary here.
% Refererence and vitae will folllow the main thesis text.
%-----------------------------------------------------------------------
% This is the beginning of the main thesis body.
% Insert Chapter or section or subsection as many as you need.

%%%%%%%%%%%%%%%%%%%%%
% Chapter 1 Introduction
%%%%%%%%%%%%%%%%%%%%%

%%%%%%%%%%%%%%%%%%%%%%%%%%%%%%%%%%%%%%
% DO NOT DELETE FOLLOWING TWO LINES! %
%%%%%%%%%%%%%%%%%%%%%%%%%%%%%%%%%%%%%%
\begin{spacing}{2.0} % double spacing
% \begin{spacing}{1.3} % double spacing
\pagenumbering{arabic}
\setcounter{page}{1}

% you start with the chapter 1

%%%%%%%%%%%%%%%%%%%%%%%%%%%%%%%%
% Chap 1. Introduction
%%%%%%%%%%%%%%%%%%%%%%%%%%%%%%%%

\chapter{Introduction}\label{chapter1}
\section{Introduction} \label{chap1:sec1}

Reinforcement Learning (RL) \cite{RL} is a method for learning an optimal policy through trial and error.
Although its theoretical foundations have been established for several decades, its practical applications were limited by various challenges.
One of the biggest challenges in reinforcement learning is extending it to continuous spaces, which leads to an increase in the dimensionality of state and action spaces.
Due to the exponential growth in the number of possible states and actions, the corresponding rise in computational complexity poses a significant obstacle to learning in high-dimensional environments.
To address this issue, traditional approaches often relied on handcrafted feature engineering to simplify the problem space.
However, designing effective features by hand is both time-consuming and domain-specific, limiting the generalizability of learned policies across different scenarios.
The emergence of deep learning addressed this issue by enabling automatic feature extraction from raw, high-dimensional inputs such as images, sensor data.
This advancement eliminated the need for manual feature design and allowed reinforcement learning agents to operate directly on raw observations.
However, applying deep learning to reinforcement learning introduced another significant challenge: the data collected by agents is highly correlated.
Unlike supervised learning, where training data is typically assumed to be independent and identically distributed (IID), RL agents interact sequentially with the environment, resulting in temporally correlated data.
This violates the IID assumption and can lead to instability and inefficient learning when training neural networks.
A major breakthrough in overcoming these limitations came with the introduction of Deep Q-Network (DQN) \cite{DQN1, DQN2} by DeepMind.
By combining deep neural networks with Q-learning, DQN enabled agents to approximate complex value functions from high-dimensional inputs such as raw pixel images.
This advancement allowed RL agent could achieve human-level performance in a variety of Atari games without relying on handcrafted features.
This success of DQN has led to significant advances in the field of deep reinforcement learning (DRL), such as AlphaGo \cite{AlphaGo} and AlphaZero \cite{AlphaZero} by DeepMind, which demonstrated superhuman performance in board games like Go, Chess, and Shogi.
In addition, OpenAI Five \cite{Five} showcased the power of DRL in complex, multi-agent environments by defeating professional human players in the real-time strategy game Dota 2.
Another notable example is Dactyl \cite{Dactyl}, a robotic hand developed by OpenAI that learned to manipulate physical objects using reinforcement learning trained in simulation and successfully transferred to the real world, highlighting progress in sim-to-real transfer for robotic control.
Despite these impressive achievements, applying reinforcement learning to real-world environments remains challenging.
RL agents typically require a large number of iterations to learn effective policies, often relying on extensive exploration to discover rewarding behaviors.
However, during this exploration process, agents may take unsafe or risky actions that can lead to catastrophic failures—particularly in safety-critical domains such as robotics, autonomous driving, or healthcare.
Moreover, even after training is complete, there is no guarantee that the learned policy will consistently behave safely, especially in unseen or out-of-distribution (OOD) environments.
In particular, transferring policies from simulation to the real world (i.e., the sim-to-real problem) can cause even greater safety concerns when learned behaviors don’t generalize well to the real world.
A key underlying difficulty is the inherent challenge of designing reward functions that reliably induce safe and desirable behaviors across a wide range of situations.

\section{Research Objective}

In this thesis, we investigate how to learn safe policies in reinforcement learning through constrained optimization techniques, focusing on State-wise Constrained Reinforcement Learning (SCRL) \cite{SCRL-survey}, which introduces cost functions to enforce state-wise safety constraints during the learning process.
Among various SCRL approaches, we examine Lagrangian-based methods due to their theoretical simplicity and empirical popularity.
This thesis analyzes the limitations of existing Lagrangian methods in the SCRL setting and empirically examines how specific design choices, including the bias initialization and the learning rate of the Lagrange multiplier network, influence both performance and safety.
We also propose a method, PPO Lagrangian Network, which extends Proximal Policy Optimization to the state-wise constraint setting using a Lagrange multiplier network.
The proposed method is empirically evaluated against existing approaches on a range of tasks from the OpenAI Safety Gym.


\section{Outline of the Thesis}

This thesis is organized as follows:

\begin{itemize}
  \item Chapter \ref{chapter2} provides background on reinforcement learning, including policy gradient methods, constrained reinforcement learning and state-wise constrained reinforcement learning. It also reviews prior work relevant to this thesis.
  \item Chapter \ref{chapter3} introduces the proposed method, PPO Lagrangian Network, which incorporates a state-wise Lagrange multiplier network into the PPO framework. The design and training procedure are detailed, along with comparisons to related methods.
  \item Chapter \ref{chapter4} presents experimental results. We first investigate the influence of key hyperparameters, such as the entropy coefficient $\alpha$, bias initialization, and learning rate of the Lagrange multiplier network. We then evaluate the performance of PPO Lagrangian network across Safety Gym tasks.
  \item Chapter \ref{chapter5} concludes the thesis with a summary of findings and discusses limitations and directions for future research.
\end{itemize}

\end{spacing}

%%%%%%%%%%%%%%%%%%%%%%%%%%%%%%%%%%%%
% Chapter 2 
%%%%%%%%%%%%%%%%%%%%%%%%%%%%%%%%%%%%

\begin{spacing}{2.0} % double spacing
%%%%%%%%%%%%%%%%%%%%%%%%%%%%%%%%
% Chap 2. Related works of action recognition
%%%%%%%%%%%%%%%%%%%%%%%%%%%%%%%%

\chapter{The chapter 2}\label{chapter2}

Write your chapter 2. Write your chapter 2. Write your chapter 2. Write your chapter 2. Write your chapter 2. Write your chapter 2. Write your chapter 2. Write your chapter 2. Write your chapter 2. Write your chapter 2. Write your chapter 2. Write your chapter 2. Write your chapter 2. Write your chapter 2. Write your chapter 2. Write your chapter 2. Write your chapter 2. Write your chapter 2. 

\begin{figure}
\centering
\includegraphics[scale=0.5]{imgs/HOG.eps}
\caption{An eps image}
\label{fig:hog}
\end{figure}

\section{A section} \label{chap2:sec1}
\subsection{A subsection} \label{sec1:1-1}

write what you did.


\begin{algorithm}[H]
  \KwData{this text}
  \KwResult{how to write algorithm with \LaTeX2e }
  initialization\;
  \While{not at end of this document}{
    read current\;
    \eIf{understand}{
      go to next section\;
      current section becomes this one\;
    }{
      go back to the beginning of current section\;
    }
  }
  \caption{How to write algorithms 1}
\end{algorithm}


Hi, hello, nice to meet you. Hi, hello, nice to meet you. Hi, hello, nice to meet you. Hi, hello, nice to meet you. Hi, hello, nice to meet you. Hi, hello, nice to meet you. Hi, hello, nice to meet you. Hi, hello, nice to meet you. Hi, hello, nice to meet you. Hi, hello, nice to meet you. Hi, hello, nice to meet you. Hi, hello, nice to meet you. 



\end{spacing}

      
%%%%%%%%
% Add more chapters if you wants
%%%%%%%%


%---------
% summary
%---------
\begin{spacing}{2.0}

\summary

This thesis studies how to train policies to satisfy state-wise safety using state-wise constrained reinforcement learning.
We focus on Lagrangian-based approaches and identify their key limitations, particularly with respect to learning stability and convergence.
To address these challenges, we propose PPO Lagrangian Network, an extension of Proximal Policy Optimization that incorporates a Lagrange multiplier network to dynamically enforce safety constraints at the state level.
Experiments on the OpenAI Safety Gym benchmark show that proposed method achieves more stable and reliable constraint satisfaction than existing approaches such as PPO Lagrangian and Feasible Actor-Critic (FAC).
We further investigate how hyperparameter choices, such as the initialization bias and learning rate of the Lagrange multiplier network, affect performance.
Our findings indicate that careful tuning of these parameters is crucial for maintaining stability under hard constraint settings.
Additionally, we demonstrate that the trained Lagrange multiplier network can be reused at test time as a risk estimator, enabling the assessment of a given state’s safety.
\end{spacing}

%-----------------------------------------------------------------------
% This is the end of the main thesis body.

%-----------------------------------------------------------------------
% Input the list of references.
\bibliographystyle{ieeetr}
\bibliography{biblist}

%%%%%%%%%%%%%%%%%%%%%%%%%%%%%%%%%%%%%%%%%%%%%%%%
% Appendix
% You can comment out if you do not need appendix
%%%%%%%%%%%%%%%%%%%%%%%%%%%%%%%%%%%%%%%%%%%%%%%%
% \begin{spacing}{2.0}
% 
%%%%%%%%%%%%%%%%%%%%%%%%%%%%%%%%%%%%%%%%%%%%%%%%
% Appendix
%%%%%%%%%%%%%%%%%%%%%%%%%%%%%%%%%%%%%%%%%%%%%%%%

%\appendixpage
\appendix

\chapter{Abbreviations}
%%% define some Abbreviations
\text{ } \text{ }  \textbf{GIST} \text{ } \text{ } Gwangju Institute of Science and Technology

\textbf{EECS}  \text{ } \text{ } Electrical Engineering and Computer Science

\textbf{YOLO}  \text{ } \text{ } You Only Live Once



\chapter{More appendix}

You can add more appendices.
% \end{spacing}

%-----------------------------------------------------------------------
% Acknowledgements
% Insert the text between \begin{acknowledgements} and \end{acknowledgements}.
% You can either write the abstract directly here or import a file using the \input command.

%%%%%%%%%%%%%%%%%%%%%%%%%%%%%%%%%%%%%%%%%%
% Acknowledgements by Korean
%%%%%%%%%%%%%%%%%%%%%%%%%%%%%%%%%%%%%%%%%%
\begin{acknowledgements}
\begin{spacing}{2.0}
%%%%%%%%%%%%%%%%%%%%%%%%%%%%%%%%%%%%%%
% Acknowledgements of the thesis in English
%%%%%%%%%%%%%%%%%%%%%%%%%%%%%%%%%%%%%%
\end{spacing}
\end{acknowledgements}


%-----------------------------------------------------------------------
% Input the curriculum vitae.
% You may add as many lines as you need using the syntax of the \item command shown below.
%-----------------------------------------------------------------------



%\input Vitae2.0.tex
%\input thesis-publications.tex

% Insert activity if you have.
%\activity
%Activity Activity Activity Activity Activity Activity Activity Activity
%Activity Activity Activity Activity Activity Activity Activity Activity

% Insert awards if you have.

%-----------------------------------------------------------------------
% This is the end of the thesis.
%
\end{document}
